\chapter{Metodologia}

\section{Visão geral da metodologia}

\section{Plataforma de desenvolvimento}
\subsection{Ambiente de testes}
\subsection{Drone terrestre como plataforma inicial}
a primeira etapa será de teste em um ambiente controlado, utilizando um drone terrestre (Carro de controle remoto) equipado com os sensores necessários para a coleta de dados. Isso permitirá validar a eficácia dos sensores antes de realizar testes em um drone aéreo. Os sensores serão configurados para registrar dados de aceleração, posição inercial e outros parâmetros relevantes durante o movimento do drone.

\section{Coleta de dados}
\subsection{Configuração dos sensores}
\subsection{Ambiente controlado de coleta}
A coleta de dados será realizada em um ambiente controlado, como um laboratório ou uma área interna segura com medidas previamente obtidas e onde o drone poderá se mover livremente. Com os dados adiquiridos, podemos comparar os resultados obtidos com os dados previamente medidos, permitindo avaliar a precisão e a confiabilidade dos sensores utilizados.

\section{Dados obtidos}
\subsection{Descrição das variáveis coletadas}

\section{Processamento dos dados}
\subsection{Transformação de coordenadas}
\subsubsection{Sequência de transformações}
LiDAR $\rightarrow$ Drone $\rightarrow$ Referencial global

\subsection{Estimativa da posição inercial}

\section{Geração da nuvem de pontos}
