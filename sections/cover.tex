\begin{titlepage}
\includegraphics[scale=0.15]{uniceub_logo}\\
\vspace{2cm}
    \begin{center}
        \vspace*{1cm}
        
        {\LARGE \textbf{Mapeamento 3D de Cavernas com Drone em Ambiente Sem GPS: Fusão LiDAR-IMU para Reconstrução}}
        
        \vspace{0.5cm}
        TRABALHO DE CONCLUSÃO DE CURSO
        
        \vspace{1.5cm}
        por \\
        \vspace{1.5cm}
            Gabriel Holsback Dantas - \url{gabriel.dantas@sempreceub.com} - RA: 22310899\\
            Rafael De Queiroz Lavoyer - \url{rafael.lavoyer@sempreceub.com} - RA: 22208760\\
        \vspace{1.0cm}
    \end{center}

    \vspace{1cm}
\noindent{
{
    {\bf Resumo:} 
    Este trabalho apresenta o desenvolvimento de uma metodologia para mapeamento tridimensional de ambientes internos utilizando um drone operando sem sinal de GPS, com aplicação voltada para ambientes confinados, como cavernas. A proposta baseia-se na integração de sensores embarcados, especialmente uma Unidade de Medição Inercial (IMU) e sensores LiDAR, combinando dados de aceleração, atitude e distância para reconstrução espacial do ambiente em forma de nuvem de pontos.
    }
}
\vfill
\textcolor[rgb]{0.5,0.5,0.5}{
    \begin{flushleft}
    { \small
    Orientador: Hudson Capanema Zaidan\\
    Turma: UN2026/07\\
    Curso: Engenharia da Computação\\
    Campus: Asa Norte\\
    Turno: Noturno
    }
    \end{flushleft}
}
      
\end{titlepage}