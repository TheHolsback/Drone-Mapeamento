\begin{titlepage}
\includegraphics[scale=0.15]{uniceub_logo}\\
\vspace{2cm}
    \begin{center}
        \vspace*{1cm}
        
        {\LARGE \textbf{Mapeamento 3D de Cavernas com Drone em Ambiente Sem GPS: Fusão LiDAR-IMU para Estimação de Trajetória e Reconstrução}}
        
        \vspace{0.5cm}
        TRABALHO DE CONCLUSÃO DE CURSO
        
        \vspace{1.5cm}
        por \\
        \vspace{1.5cm}
            Gabriel Holsback Dantas - \url{gabriel.dantas@sempreceub.com} - RA: 22310899\\
            Rafael De Queiroz Lavoyer - \url{rafael.lavoyer@sempreceub.com} - RA: 22208760\\
        \vspace{1.0cm}
    \end{center}

    \vspace{1cm}
\noindent{
{
    {\bf Resumo:} 
    Este trabalho apresenta o desenvolvimento de uma metodologia para mapeamento tridimensional de ambientes internos utilizando um drone operando sem sinal de GPS, com aplicação voltada para ambientes confinados, como cavernas. A proposta baseia-se na integração de sensores embarcados, especialmente uma Unidade de Medição Inercial (IMU) e sensores LiDAR, combinando dados de aceleração, atitude e distância para reconstrução espacial do ambiente em forma de nuvem de pontos. 

O sistema utiliza princípios das equações do movimento newtoniano para estimar a posição do drone ao longo do tempo, realizando integrações sucessivas da aceleração após a correção das inclinações do veículo. Para isso, as acelerações medidas no referencial do corpo são convertidas para o referencial do mundo por meio de matrizes de rotação baseadas em roll, pitch e yaw, permitindo remover corretamente a influência da gravidade e obter o deslocamento real do equipamento. 

A partir da estimativa de posição do drone, os pontos medidos pelo LiDAR passam por transformações entre referenciais — LiDAR e drone possibilitando a determinação da posição global de cada ponto capturado. Esse processo resulta na construção de uma nuvem de pontos capaz de representar o ambiente tridimensionalmente com maior precisão. 

Além da modelagem matemática, o projeto contempla a definição da arquitetura de hardware embarcado, incluindo microcontrolador, sensores, módulos de armazenamento, sistema de alimentação e estrutura física, considerando critérios de peso, consumo energético e autonomia necessários para execução do mapeamento. 

Como resultado esperado, o trabalho busca demonstrar a viabilidade de uma solução de baixo custo e independente de infraestrutura externa para mapeamento 3D, contribuindo para aplicações em exploração, inspeção e documentação de ambientes onde sistemas tradicionais de posicionamento não estão disponíveis.
    }
}
\vfill
\textcolor[rgb]{0.5,0.5,0.5}{
    \begin{flushleft}
    { \small
    Orientador: Hudson Capanema Zaidan\\
    Turma: UN2026/07\\
    Curso: Engenharia da Computação\\
    Campus: Asa Norte\\
    Turno: Noturno
    }
    \end{flushleft}
}
      
\end{titlepage}